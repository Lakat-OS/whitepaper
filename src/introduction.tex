
\section{Introduction}

With the vast amount of data structures, of query and storage systems, of versioning and networking tools and of large language models, one may engineer publishing systems by posing certain requirements that give rise to a different and arguably more collaborative, efficient and healthy academic culture. This approach can be contrasted with an incremental adjustment of the existing system, which in many quantitative sciences is called the greedy approach. We propose one such architecture around the available technology that we call \textit{Lakat}.
%, whilst acknowledging and demarcating it from other proposals. 
Lakat is a distributed database with a local peer-review concensus layer. The system serves as a permissionless continuous integration solution for collaborative research and one may conceptionally think of Lakat as a peer-to-peer  version of wikipedia with a branch structure similar to git and a peer review system.
Our starting point is a set of eight core requirements, that we posit for a publishing system:\\

\indent \textbf{1. Open} -- 
 Content and code base\footnote{Here we refer to the code base of any client implementation.} should be accessible freely\footnote{Internet service providers are not free. So we refer here to additional charges.}.\\
\indent\textbf{2. Permissionless} --
 No one should be barred from contributing.\\
\indent\textbf{3. Pluralistic} -- No monopoly on research opinion.\footnote{This is not not necessarily the same as ``No single source of truth''.}\\
\indent\textbf{4. Process-oriented} -- Emphasizing the process rather than an outcome.\\
\indent\textbf{5. Conflict-oriented} -- Making conflicts a feature rather than a bug.\\
\indent\textbf{6. Curatable} -- Making the organization of the content part of the output.\\
\indent\textbf{7. Sustainable} -- 
 Data and compute resources should be low and reuse of fragments encouraged.\\
%  \item \textbf{soft constraints}
\indent\textbf{8. AI friendly} -- Allowing all kind of entities to contribute, indivduals, groups or AI agents.\\
% We discuss the elements of Lakat in Section \ref{sc:elements}.
% followed by  as well as typical user journeys and attack vectors.

% We propose a distributed database with a local peer-reviewed concensus layer. The system serves as a continuous integration and continuous development (CI/CD) solution for collaborative research and one may conceptionally think of Lakat as a decentralized version of wikipedia with a branch structure similar to git and a peer review system.


% The assumptions in a core are constituting in the sense that the research programme would cease to exist without them. 
% Rather than saying -- as one might imagine -- that the science--predicate of research programmes relies on modifications of this hard core, he instead endows them with dogmatic immutability, disallowing any hypothetical modifications of them. In a quest for guarding this sacred hard core a research programme can only make amendments to the protective belt that ought to divert possible attacks to the core. An attack cosists in a conflict of the theory with empiric evidence. 

% Science and publishing of scientific output appears to be going through a crisis. 
% ... interoperable. research papers are little blobs. 
The research paper as the gold standard of publisicing research output poses several threats to the overall scientific endeavour. It is a relict from the times where the printing press had been the latest innovation and where the channels for communicating had a large latency.  We mention six issues associated with the paper-formatted research output that are addressed by Lakat:
\begin{itemize}

 \item It incentivizes the creator(s) of scientific output to withhold preliminary results or results that are either not significant or at odds with a hypothesis. Even if there are significant results\footnote{These may be perceived as significant or later recognized as significant by the community}, they may not meet the eye or mind of other creators or consumers of scientific output until the entire paper has been published. It may then even take of the order of tens of months for the paper-formatted research output to be accessible, which is particularly problematic for impactful research. Thus the process of building on top of previous work and of critical engagement is hindered and in the best case deferred.
 
\item It incentivize creator(s) to wrap minor changes into the costume of an entire research paper, reusing a possibly templated introduction over and over again. 

\item The output is but a polished snapshot of a process, an inorganic blob "data structure". The process of reaching a result or of not reaching it as well as the review process are generally not part of the output and not naturally representable in the rigid paper-format. The process often doesn't stop with the paper-publication, but continues thereafter and it requires awkward hacks in the form of addenda, corrections or new paper-formatted versions to account for changes.

\item It creates rigid and isolated islands of content, disregarding potentially conflicting or agreeing intersections. Papers address these intersections with citations that are often placed in an unspecific context, and tend to reference an entire paper or body of work rather than a particular part. These intersections between different scientific outputs are not only constrained to citations, but entire paragraphs such as introduction or method sections are often simply replicated from previous papers. Thus, making conflicting or agreeing intersections a manifest part of the data structure can overcome the hacky fixes and shortcommings of the paper-format.

\item 
The question of who contributed how much to a research output is often a source of conflict among researchers. A process-oriented publication system facilitates the tracking of contributions and may reduce the cases of unjust allocation of contributorship. In paper-formatted publications the contributorship is proxied by a negotiated ordered list of co-authors, which cannot capture contributions and inevitably leads to unjust allocations. 

\item The effective barring of potential contributors in paper-formatted research does not increase the level of scrutiny, creativity and quality of the output. On the contrary, maybe another set of eyes can add insights or expand on the results. Why should the self-declared co-authors be in the best position to conduct the research? The fear for the theft of ideas is mostly inherent to bulk-publications and less to process-based research output. 
\end{itemize}

Apart from the abovementioned problems with paper-formatted research, Lakat may also be instrumental for solving other problems with scientific publishing such as the exploitation of scientists regarding their review services and production of output. Even though Lakat does not address this directly, it does provide a base layer to build a system of incentives on top of it. 

\subsection{Related Work}

There are various solutions that have been proposed to improve the process of science publishing with respect to transparency, review, ownership, decentralization, collaboration, openness or fairness. We exhibit proposed solutions and their benefits or shortcomings. Since Lakat sits at the intersection of branchable version control (c.f. git\cite{}\remark{cite git}), large collaborative enyclopedias (c.f. wikipedia\cite{}\remark{cite wiki}) and peer-to-peer (c.f. humans \cite{}\remark{cite humans decentralization}) protocols (c.f. urbit \cite{}\remark{cite urbit} or ipfs\cite{}\remark{cite ipfs}), we will focus on solutions in that general triangle.

In 2006 the platform Scholarpedia \cite{}\remark{find source} was launched. It is a wiki-based format with a peer review layer, where institutional affiliation is required for contributon. It is thus integrating a scholarly component into wikipedia. The required affiliation is also one of the drawbacks of this solution, since some potential contributors are barred. Furthermore, the authors of an article are either chosen or elected. This to our mind has two further problems. First, it raises the question who elects those that elect. Second, the collaborative dimension of wikipedia is lost. In contrast Lakat -- like wikipedia -- retains the permissionless so that no one is barred from editing or from proposing pull requests to change content (see Section \ref{sc:elements} for details). In 2007 the Citizendium fork of the English wikipedia launched \cite{} \remark{find source} with the objective to add a quality assurance layer on top of wikipedia. The concept of approved aricles played an important role. However, who approves the articles. What happens to subsequent changes? Would they have to be approved again or does the approval yield a sort of finality for the manuscript?
Another wiki-formatted solution is the Manubot platform \cite{himmelstein2019open}, which allows for collaborative preparation of research articles that can then be sent to peer-reviewed journals. Manubot, however is not a publication platform itself but aids the collaborative process of reaching a classical publication. 

There are also many attempts to put part of the existing publishing logic onto a cryptographically secured distributed ledger. Everipedia \cite{} \remark{find source}  was a fork of wikipedia. They have also tried to build a quality assurance system on top of it using reputation tokens that can be staked and potentially lost in the process of edits, thus leveraging distributed ledger technology. So instead of tokenizing ownership of edits, they tokenized reputation. Those tokens were deployed on a blockchain (EOS and later Polygon). The project has been archived. Orvium \cite{}\remark{find whitepaper link} on the other hand aims to put submission of manuscripts, revisions and publications onto a blockchain or at least have them stored using some decentralized storage provider. Unfortunately it is not evident who stores what, how and where. There is for instance not much information about whether they are creating a dedicated blockchain or use an existing one.\remark{They also launched a designated token to further raise captial for their undertakings.} 
The Scienceroot project \cite{}\remark{find source} was launched in 2018 with the intend to create an on-chain economy around the publishing system using a reward token called Science Token (ST) which is deployed on the Waves blockchain. They also created or attempted to create an academic journal that ties into their economy. Pluto \cite{}\remark{find source} is a blockchain-based platform for academic publishing that supports peer review, open access and micropayments. ARTiFACTS \cite{}\remark{find source} is a project that aims to create a blockchain-based platform for scholarly research that enables researchers to create a permanent, time-stamped record of their various items that support their research such as data sets, images, figures etc. 
PubChain \cite{}\remark{find source} is a project that aims to create a decentralized open-access publication platform that combines a funding platform with decentralized publishing. Like Scienceroot, it has its own token coincidentally also called Science Token (ST), which is used to exchange funds, store articles on ipfs and store their content identifiers on the blockchain. They also plan to integrate crowdfunding through their marketplace. TimedChain \cite{}\remark{find source} is a project that aims to create a blockchain-based editorial management system that organizes manuscripts by publishers, authors, readers and other third parties. EUREKA \cite{schaufelbuhl2019eureka} is a project that aimed to create a blockchain-based peer-to-peer scientific data publishing platform with peer review, open access and micropayments. It was developed by the team behind ScienceMatters, an existing open access publisher that conducts triple-blind peer review. EUREKA also aimed to provide a blockchain-based rating and review system that allows readers to evaluate the impact of published articles. It is, however, not any more maintained. 
The Open Science company Desci Labs is developing a project called Desci Nodes \cite{descilabs}. Similar to Scienceroot, DeSci Nodes is a tool for creating research objects, which are a type of verifiable scientific publication that combines manuscripts, code, data, and more into a coherent unit of knowledge.  
The 2018 "nature index" article \cite{brock2018} entitled "Could Blockchain Unblock Science?" focusses on the question of how blockchain could be used to improve the process of current science publishing. Brock also mentions that data edits could be made permanently visible, which aludes to the idea of securing continuous editing in an immutable and consensual manner. He also developed and deployed the Frankl, which is an open source blockchain-based publishing platform \cite{brockopenscience2018}. 

When developing solutions for academic publishing, blockchain technology seems appealing, because it yields effectively immutable globally agreed data in an open and transparent way without the need for a single source of trust. However, one must not fall into the falacy of searching nails for a hammer. At the heart of the blockchain paradigm lies the idea of a consensus about a global unique truth. This is a very useful technology for fiat (e.g. printed money or cryptocurrency), which exists through a global consensus. However, research output is not a fiat currency. It is subject to conflicting theories, opposing views and possibly irreconcilable results. All of those drive the continuous process that is science. One may build solutions on top of a blockchain to allow for potentially conflictual editing, but this is not what it was designed for. Instead we suggest to make Lakat a base layer that satisfies the requirements for a publication system by design.
   
There are also a range of solutions that attempt to decentralize version control systems or anchor them in a blockchain. One of the most prominent examples of a decentralized version of a version control system with branches is git-ssb \cite{gitssb}. It is based on the secure scuttlebutt protocol \cite{scuttlebutt} and allows for distributed version control. The Radicle protocol \cite{radicle} is a peer-to-peer network for code collaboration. It extends git with a networking protocol called Radicle Link. The project is governed through the RAD token, which is deployed on the ethereum blockchain.
Both of the above are great tools for decentralized version control of code. However, they are not designed for the purpose of academic publishing. 
 Another project that explores ways to decentralize the storage of versioned data is ceramic \cite{ceramic}. It is a decentralized network for managing mutable information. It is based on the idea of streams, which are append-only logs of JSON objects. The streams are anchored in a blockchain, which is used as a global ordering mechanism. The streams are stored in a decentralized storage network.  

With the onset of large language models (LLMs) and AI agents that are capable of statistically extrapolating from a vast set of existing resources we are entering an era where some portions of the scientific research process can and should be outsourced to these models. Autonomous agents should be able to take part in the process of scientific discovery. The impact and power of AI-aided research can be seen in multiple forms and with the recent advances in the field of AI and the availability of large amounts of data its value for the scientific community only grows.  

% \cite{tenorio2019towards}
% \cite{boiko2023emergent}

\subsection{Imre Lakatos}

The entire architecture of Lakat is heavily inspired by concepts developed by the Hungarian philosopher Imre Lakatos. In an attempt to contribute to the demarcation problem\cite{} that was prominent in the field of philosophy of science during Lakatos' times, he developed the concept of a \textit{research programme}\cite{}, which is also called \textit{Lakatosian research programme} to avoid confusion with the colloquial use of the former term. The demarcation problem asks about the criteria which tell science apart from "pseudo-science".  Lakatos develops his theory on the grounds of a process-oriented account of science. So rather than saying that this or that monolitic bulk of work or set of statements is or is not scientific, he posits that this distinction can only be made on the grounds of processes of theoretical amendments to an existing corpus of statements. He distinguishes progressive and degenerative amendments depending on whether they strengthen the programme's predictive power. For Lakatos a research programme consists of a \textit{hard core}, which is a set of constituting assumptions, axioms as it were, that capture the essence of a research endeavour and a \textit{protective belt} of auxiliary hypotheses. The key ideas that the Lakat-architecture takes from the concept of the Lakatosian research programmes are threefold: 1) The pluralism of various research undertakings. 2) The process-orientation 3) The distinction between a core and a protective belt. At the heart of these foundational concepts lies the idea that science lives through arguments, differences and discourse. The input of Lakatosian concepts into Lakat can then be described as follows: A research programme corresponds to a branch or a set of branches to which researchers contribute changes or ammendments. There is no single master branch, but rather every research programme has its own branch or set of branches. Conflicts with other branches or even within the same branch are an important aspect of Lakat and can be the source of progress (c.f. progressive ammendments in Lakatosian research programmes). A programme can maintain a set of feature branches that support the core branch. These side branches behave like a protective belt.  
% 

\subsection{Overview}
% 
% \section{Requirements}
% 
% In this section we provide a list of high-level requirements that we wish to be satisfied by a publication system. We then consider three existing publishing systems in view of those requirements:
% 
% \indent \textbf{1. Open} -- 
%  Content and code base\footnote{Here we refer to the code base of any client implementation.} should be accessible freely\footnote{Internet service providers are not free. So we refer here to additional charges.}.\\
% \indent\textbf{2. Permissionless} --
%  No one should be barred from contributing.\\
% \indent\textbf{3. Pluralistic} -- No monopoly on research opinion.\footnote{This is not not necessarily the same as ``No single source of truth''.}\\
% \indent\textbf{4. Process-oriented} -- Putting the curation on par with the content.\\
% \indent\textbf{5. Conflict-oriented} -- Making irreconcilabilities manifest.\\
% \indent\textbf{6. Curatable} -- Putting the curation on par with the content.\\
% \indent\textbf{7. Sustainable} -- 
%  Data and compute resources should be low and reuse of fragments encouraged.\\
% %  \item \textbf{soft constraints}
% \indent\textbf{8. AI friendly} -- AI should be able to interact with the system.\\
% 
% The classical route to publishing is through sending a manuscript to the editors of a peer reviewed journal. The editor(s) then search(es) for reviewers willing to review the manuscript. Is this open? In most cases the article as well as the review, the code base and the data -- if there is any --  are locked behind pay walls, i.e. not accessible publicly. Some journals have various open access options, that allow the abovementioned parts of the publication to be accessible, albeit at a high price for the author both in terms of money and in terms of transferral of rights to the publisher. In principle the system is permissionless, except for those cases where publication fees apply and/or membership in a recognised institution is required. This system is not particularly monopolous, at least prima facie. The de facto sovereignty over what is an exceptable contribution rests with the editor(s) of the journals, but given enough funding new journals can form. The process of the research as well as the review are not well mapped in this system. There is no focus and in particular no build-in focus on resolution of conflicting results. Curation is left to the editors. 
% 
% Wikipedia\footnote{Here we refer not only to the software implementation, which at the time of writing is version 1.39 of mediawiki, publicly accessible via wikipedia.org (April 2023), but also the body of editors and maintainers.} is an open database. Both its code-base as well as its content can be viewed with access to a non-censored internet connection. A large part of the database can be edited permissionlessly, even without an account. Every article has one current version, often with a rich history of older versions and vivid discussions around edits. At times fierce editing battles can be observed and have been subjected to scientific studies \cite{dedeo}. 
% The possibility to express different views on a matter in the form of versions as well as the discussion adds a pluralistic element, even though the existence of a single current truth in the form of the latest version of an article deducts from that. the multilingual featuer on the other hand makes wikipedia more pluralistic. 
% 
% Programmable blockchain is an open database. Is is often referred to as bein permissionless and that is true for most blockchains in the sense that no third party controls read or write access to it.
% 
% 
%  Making a manifestly pluralist system not only prevents the dictatorship of one academic credo over another, but also allows for conflicting opinions to develope and hopefully to resolve.
 

% \section{Elements of the Publishing System}
% \label{sc:elements}

With Lakat we propose a manifestly pluralistic, process and conflict-oriented architecture for the continuous integration of publications with a primary use case of research publications. In this way Lakat becomes a living document. At its core the architecture consists of a linked data structure that resembles a DAG, where the key objects are branches. This data structure facilitates collaborative work in much the same way as git does. Branches may be thought of as the analogue of a journal in traditional publishing. The role of journal editors is covered largely by branch contributors. Branches are chains of blocks that contain submissions. Addition of another block happens via a proof of peer review, where the peers are the contributors to that branch. In that sense branches resemble blockchains with blocks consisting of submitted changes instead of transactions. As a consensus mechanism we discuss a solution that combines a proof-of-review at branch-level, a local (i.e. involving just branch-contributors) consensus rather than a global one, with a new finality gadget called Lignification. The review process is open. In a first version of Lakat the identities of the reviewers and the creators of the reviewed content are disclosed, however we wish to migrate to a weak\footnote{Weak is to be understood in the sense that both parties may choose to reveal their identity.} form of a double-blind protocol leveraging zero-knowlege proofs, where each party may reveal their identity.
Data is content-addressable and conforms to the ipld multihash format. Storage is handled by a networking component in Lakat which delegates the bulk of data storage to a selection of other storage providers, including decentralized storage networks such as ipfs, storj and others. This improves resiliance and longevity. 

In the following we discuss the individual elements of the proposed system and highlight their interaction.

% In the following we discuss the individual elements of the proposed system and highlight their interaction. We start with the data structure in section \ref{sc:datastructure}, which is the core of the system.