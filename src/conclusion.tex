\section{Conclusion}
\label{sc:conclusion}

The contributions presented in this paper are threefold. First we propose a process- and conflict-oriented academic publishing system that is based on a peer-to-peer network architecture and supports continuous integration across multiple branches. Second, we propose a new consensus mechanism for branched, permissionless systems, called Proof of Review. Third, we propose a new finality gadget, called Lignification, that is specifically designed for branched, permissionless systems. 

Regarding the first contribution of an architecture for an academic publishing system we provided a list of high-level requirements for such a system. We then argue that paper-based publishing has major shortcomings including the incentivization to withhold preliminary results, the tendency to wrap minor changes into an entire research paper, the lack of representation of the research process in the output, the creation of isolated content islands, the difficulty of tracking contributions and the barring of potential contributors. Based on that we propose an architecture that meets the requirements and addresses these shortcomings. It is composed of a cryptographically linked data structure that is kept in a distributed key-value database whose entries are stored, updated and communicated using a peer-to-peer network protocol. Storage is partly outsourced to other protocols. The central data objects are branches and data buckets. All research content as well as updates and context information is stored in buckets. Branches on the other hand are cryptographically linked chains of changes to this underlying content. These changes are bundled up into submits, which are analogous to Git-commits. Every branch has its own staging area, where any type of branch interactions are waiting to be included. We distinguish three types of branches: proper branches which behave like production branches, twigs which are used like feature branches and sprouts which are temporary auxiliary branches. All branches are controlled by branch contributors. We distinguished four types of branch contributors: Content contributors who can prove to have submitted content to a given branch, review contributors who can prove to have pushed reviews to the branch, token contributors who can prove to have deposited funds into the branch and storage contributors who can prove to store data of that branch. Branch modifictions happen via submits through a local consensus and finality mechanism amongst the branch contributors. In principle the Lakat specification does not require a particular choice of these mechanisms.

We put forth a particular local consensus mechanisms for proposing changes in branched, permissionless systems. This constitutes the second contribution. The mechanism distinguishes between feature and production branches. Using Lakat terminology these are twigs and proper branches. Amendmends to both types of branches are made through submits. Twigs ought to be used for small changes and quick iterations. Adding a submit requires a majority vote amongst the branch contributors, which in the case of a single contributor leads to no consensus rule at all. Proper branches on the othe hand are used for larger changes. Amendments to proper branches involve merge submits, but the process is more akin to adding a block to a blockchain. First a pull-request is sent by a requesting branch to the staging area of the target branch. Any content contributor from the target branch may send a review commitment to the requesting branch and can participate in the review process through its role as a review contributor. Once a request has been reviewed and approved by a set number of reviewers a formally valid merge submit can be formed. Any contributor of the target branch can propose the amendment of the merge submit to the proper branch. We call this process the Proof-of-Review (PoR).

Finally we propose a mechanism to deterministically decide which proposed merge submits are eventually ammended to the target branch. We put forth a finality gadget called "Lignification" which we presented as our third contribution. In order to decide which of the proposed merge submits become the head of the proper branch, we proposed a process using temporary proto-branches called sprouts. Every merge submit bears in it the possibility of becoming either the head of the proper branch or the head of forked branch. That is why we proposed to wrap the proposed merge submits into sprouts. There could be multiple proposed merge submits. In the lignification method there is a possiblity for the branch contributors to issue vetos and subsequently votes for case of multiple contesting merge submits. The lignification times and engagement times are respectively reserved for vetoing and voting. A broadcasting buffer is used to accommodate for network latency. 

We also mentioned Lakat's potential for interfacing with existing software and protocols, such as mediawiki, IPFS, Ceramic, Urbit, git, radicle, and Polkadot. Moreover, we drew a possible path to onramp users such as journals and scientist to Lakat. This further enhances its adaptability and integration capabilities, which is crucial for the system's adoption and growth, ensuring that it can evolve alongside technological advancements and changing academic needs.

In a next step we are inviting anyone to collaboratively build client software for Lakat. We are currently in the process of building a Rust-client, called Lakat-OS. Please visit our \href{https://github.com/Lakat-OS}{Lakat Github Repository}.

% In conclusion, Lakat represents a significant step forward in the evolution of academic publishing. By prioritizing collaboration, process-orientation, and pluralism, it has the potential to transform the way research is conducted and disseminated, ultimately contributing to a healthier and more productive academic culture. As the system continues to develop and adapt, it is anticipated that it will play a pivotal role in shaping the future of academic publishing.