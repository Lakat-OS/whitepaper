\section{Conclusion}
\label{sc:conclusion}

The contributions presented in this paper are threefold. First we propose a pluralistic process- and conflict-oriented publishing system that is based on a peer-to-peer network architecture and supports continuous integration across multiple branches. Second, we propose a new consensus mechanism for branched, permissionless systems, called Proof of Review. Third, we propose a new finality gadget, called Lignification, that is specifically designed for branched, permissionless systems. 

Regarding the first contribution of an architecture for academic publishing we provide a list of high-level requirements for such a system. We argue that paper-based publishing has major shortcomings including the incentivization to withhold preliminary results, the tendency to wrap minor changes into an entire research paper, the lack of representation of the research process in the output, the creation of isolated content islands, the difficulty of tracking contributions and the barring of potential contributors. We then propose an architecture that meets the requirements and addresses these shortcomings. It is composed of a cryptographically linked data structure that is kept in a distributed key-value database, whose entries are stored, updated and communicated using a gossiping peer-to-peer network protocol. Storage is partly outsourced to other protocols. The central data objects are the branches and data buckets. All research content as well as updates and layout information is stored in buckets. Branches on the other hand are cryptographically linked chains of changes to this underlying content. These changes are bundled up in a submit object, the analogue of a git commit. Every branch has its own staging area, where any type of branch interactions are waiting to be included. Branches are controlled by branch contributors. We distinguished four types of branch contributos: Content contributors who can prove to have submitted content to a given branch, review contributors who can prove to have pushed reviews to the branch, token contributors who can prove to have deposited funds into the branch and storage contributors who can prove to store data of that branch. These contributors are not stored in the branch object.



chained  be  a research project and is composed of \emph{commits}, which represent contributions to the project. The data structure is designed to be flexible and extensible, allowing for the integration of existing reputation systems and incentive structures or the development of new ones. The network protocol provides a robust foundation for the system's growth and evolution, ensuring that it can adapt to technological advancements and changing academic needs. Moreover, the system's design allows for a seamless transition from the traditional publishing model, making it a viable solution for the current academic landscape. 


The data structure is designed to be flexible and extensible, allowing for the integration of existing reputation systems and incentive structures or the development of new ones. The network protocol provides a robust foundation for the system's growth and evolution, ensuring that it can adapt to technological advancements and changing academic needs. Moreover, the system's design allows for a seamless transition from the traditional publishing model, making it a viable solution for the current academic landscape. 

, namely branches,   a  around the concept of branches  peer-to-peer 

we presented a 


In this paper we have presented Lakat, a novel base layer for a publishing system that fosters collaboration, pluralism and permissionless participation. Drawing inspiration from the philosophy of Imre Lakatos, Lakat is designed as a peer-to-peer process- and conflict-oriented system that supports continuous integration across multiple branches. This architecture provides a robust foundation for the  integration of existing reputation systems and incentive structures or the development of new ones. Secondly, we propose a new consensus mechanism, called Proof of Review, which ensures the integrity and quality of the content while promoting active participation from the community. Lastly, we present Lignification, a new finality gadget specifically designed for branched, permissionless systems. Lignification provides a deterministic way to determine the consensual state in these systems, ensuring the system's robustness and reliability in handling complex scenarios where multiple contributors may be proposing changes simultaneously. Together, these contributions aim to provide a convenient starting point to tackle some of the issues in traditional paper-formatted publishing of research output. By prioritizing collaboration, process-orientation, and pluralism, Lakat aims to improve the way research is conducted and disseminated and ultimately hopes to contribute to a healthier and more productive academic culture.


This paper has presented Lakat, an architecture for a publishing system that aims to contribute foster collaboration, efficiency, and openness. Drawing inspiration from the philosophy of Imre Lakatos, Lakat is designed as a decentralized, pluralistic, and process-oriented system that addresses the limitations of traditional paper-formatted research output.

% The data structure and network protocol of Lakat, combined with its approach to participant roles and identity management, provide a robust foundation for a new era of academic publishing. The system's design allows for a seamless transition from the traditional publishing model, making it a viable solution for the current academic landscape.

% Moreover, Lakat's potential for interfacing with existing software and protocols, such as mediawiki, IPFS, Ceramic, Urbit, git, radicle, and Polkadot, further enhances its adaptability and integration capabilities. This flexibility is crucial for the system's adoption and growth, ensuring that it can evolve alongside technological advancements and changing academic needs.

% In conclusion, Lakat represents a significant step forward in the evolution of academic publishing. By prioritizing collaboration, process-orientation, and pluralism, it has the potential to transform the way research is conducted and disseminated, ultimately contributing to a healthier and more productive academic culture. As the system continues to develop and adapt, it is anticipated that it will play a pivotal role in shaping the future of academic publishing.