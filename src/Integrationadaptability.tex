\section{Integration and Adaptability}
\label{sc:integrationadaptability}


\subsection{Onramping}
\label{ssc:onramping}

One of the objectives of Lakat is to transition academic publishing from a paper-formatted system to a cryptographically secure, collaborative and pluralistic system that allows for the continuous integration of research output. In order to achieve this objective, we believe that a transition should be as seemless as possible. The publishing system with isolated paper-formatted publications and intransparent review processes is an edge case of Lakat, an unsustainable and hacky one yet sufficient for onramping. We describe in which sense this is the case and how a transition could be achieved.

We can imagine a scenario for Lakat with a set of isolated branches. Each branch is controlled by a single legal entity, namely an academic journal. The academic journal is the content contributor, the storage contributor and the token contributor all in one. When a hypothetical researcher, say Alice (AI or human), wants to publish a paper, she has to send it to the journal. The branch that the journal  controls is simply the indexed collection of articles that have been submitted, respectively chained together cryptographically. For a journal to transition its content to such a branch state is anywhere between immediate -- by pointing the head of the branch to the storage locations of all content -- or a matter of running a script that creates a submit for each accepted article retrospectively.  Each paper is stored on a journal-controlled server, thus making the journal the sole storage provider. By adding the submissoin to the journal branch, the journal becomse the sole content contributor and retains all the rights of the contribution. The contribution is no longer owned by Alice at all. In this hypothetical oligarchic abberation of Lakat, a contribution is a submit with a single data bucket containing the paper. In summary, there is a way to map the classical publishing system into Lakat. Depending on the openness and licensing it might be difficult to either access or modify the content, but at least there is an entry point for the conversion.
Why is this unsustainable?  Given the design of Lakat, this branch would quite naturally undergo diversification through forking. At some point a researcher may create a branch routed in that journal branch, which is but a click. Maybe the incentive structure provided by the journal is soo strong that authors are willing to transfer all the rights to the journal voluntarily, but given Lakat's inherent ease of branching it will be a matter of time until there a diversification is to be expected.



\subsection{Interfaces}
\label{ssc:interfaces}

We envision Lakat as a base layer for an open, pluralistic and collaborative publication system that progresses through continuous integration. As a base layer we strive to rely only on a bare minimum of other software and aim to have interface with for existing software or protocols. Here we provide an overview of the protocols and software that we plan to build upon or interface with.

We would like to interface with mediawiki. Mediawiki is an evolving database schema with a php frontend that allows for the creation of knowledge databases such as wikipedia. There are various ways how Lakat could interface with mediawiki. The weakest form requires the conversion of the data contributions in Lakat to database entries in mediawiki. A stronger form converts also contributions to mediawiki into Lakat contributions. Regarding storage we aim to stay agnostic and leave the storage protocol as a configurable option. As options we consider IPFS, Ceramic (which is built on top of IPFS and anchored in Ethereum) and Urbit (lineDB). Regarding the token layer we aim to be agnostic here as well. Since this is an optional feature it is left to the branch contributors to decide and merge updates on their token transactions into their branch. We do recomment to deploy tokens on Polygon though and plan to integrate this into the pipeline.

Regarding version control we would like to reduce the complexity of branch operations to a bare minimum in order to avoid security threats and reliance on other protocols. For the local consensus mechanism we believe that a heavily reduced set of operations is favourable. Nevertheless we would like to interface with existing version control systems such as git or radicle. We would like to interface with them in order to allow for the conversion of git or radicle branches into Lakat branches.

We would also like to allow for new branches to be turned into parathreads in Polkadot. This would allow for the integration of Lakat into the Polkadot ecosystem. To this end one would need to create a pipeline to spin up a new parathread using Substrate together with a custom consensus protocol, namely the Lakat protocol. One would have to set the \textit{BlockImport} to a custom way of importing new submits into the key-value database. \textit{SelectChain} handles the finalization mechanism and would need to be set to the Lignification mechanism (See Subsection \ref{ssc:lignification}). One would also need to set the proof-of-review mechanism (See Subsection \ref{ssc:por}) in the \textit{Environment} option of the Substrate runtime.
