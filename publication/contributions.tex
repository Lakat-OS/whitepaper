

\section{Participants}
\label{sc:participantsandnodes}

\subsection{User Identity}
\label{ssc:accounts}

We propose to use an identity management system that ties in with some of the existing decentralized identity schemes and allows for the integration of multiple signing methods. For the first version of Lakat we propose to use 3id as the identity management system. 3id is a W3C compliant decentralized identity management system that allows for the integration of multiple signing keys. The identifier for 3id within the did system is did:3. It relies on a mutable document type in the ceramic network, called a stream. In the future we would also like to reduce identity to the ability to prove the submission of content without exposing further information about the identity using zero knowledge technology without a trusted setup. This would allow for a double blind review system. In order to publish content or send messages into the network a user needs to have an identity, which at this point is a did:3 identity registered in the ceramic network. The private keys in the did are used to sign messages and to prove authorship of content. The keys are also used to sign messages that are used to propose new states of a branch.

\subsection{Contributor}
\label{ssc:contributors}
Every branch has \textit{contributors}, or rather contributors have branches. A contributor is an account that can prove to have contributed to a given branch. There are four types of contributors for any given branch: \textit{content contributors}, \textit{review contributors}, \textit{token contributors} and \textit{storage contributors}. A content contributor can prove to have submitted to the branch. A review contributor is someone who can prove to have pushed reviews to the branch (see Subsection \ref{ssc:por} for information on proof of review). A token contributor is someone who can prove to have deposited funds into the branch. A storage contributor is someone who can prove to store data of that branch. Being a contributor means that you have to prove your contribution for the submits from the root submit of the branch till the current stable head.
How does the set of contributors change during a merge? What is the relation between the contributors of two branches before the merge and after? When the belt branch is merged into the core following a pull request, then the new set of contributors is simply the union of the two branches (see Subsection \ref{ssc:branch} for the terms core and belt). That holds for all contributor types. When there is no pull request preceding the merge the contributors of this branch are unaltered.
The main idea behind the concept of contributors is derived from the mutability, governance and autonomy of branches. Branches can only be modified by their contributors. This attempts to preempt attacks on branches.


\subsection{Contribution}
\label{ssc:contribution}

A contribution is any type of interaction with the branch. There are four types of contributions: \textit{submit}, \textit{review}, \textit{token} and \textit{storage}. A submit is a contribution that adds data to the branch. A review submit is a contribution that adds a review to a branch. A token submit is a contribution that adds a proof of a token creation, token mint or token transfer to the branch. A storage contribution is a proof about the storage of data contained in the branch, i.e. pointed at by submits of the branch. A contribution is always associated with a branch and a contributor. In the first minimal viable version of Lakat, we are planning to use zk-STARKs as a proof system. This is due to the fact that zk-STARKs do not require a trusted setup. We use Cairo programmes to generate proofs and point to their verification. The proof of contribution is a hash of the zero-knowledge proof of the contribution and its verification.

When a branch request is sent into the network it is being routed using the Kademlia protocol \cite{maymounkov2002kademlia} to the contributors of the corresponding branch. The backlog of requests is being shared and continuously broadcasted and updated by storage and content contributors using the libp2p library \cite{libp2p}. If the request has a payload that ought to modify the branch state, the receiving node checks the proof of contribution. If the proof is valid the node adds the contribution to the branch. If the proof is invalid the branch rejects the contribution. If the contribution is a submit and the submit is not valid it cannot be included in the branch. 